\documentclass{beamer}
\usepackage[utf8]{inputenc}
\usepackage{graphicx}

% Tema
\usetheme{Madrid}

% Informações do documento
\title{O Algoritmo KNN: Um Guia Completo}
\author{Seu Nome e Outros Autores}
\date{\today}

\begin{document}

% Slide de título
\frame{\titlepage}

% Slide 1: Introdução
\begin{frame}
\frametitle{Introdução}
\framesubtitle{O que é o KNN?}
\begin{itemize}
    \item Um classificador de aprendizado supervisionado não paramétrico.
    \item Utiliza a proximidade entre pontos de dados para fazer classificações e previsões.
    \item É um dos classificadores mais populares e simples em machine learning.
\end{itemize}
\end{frame}

% Slide 2: Funcionamento
\begin{frame}
\frametitle{Funcionamento}
\framesubtitle{Classificação e Regressão}
\begin{columns}
\column{0.5\textwidth}
\textbf{Classificação:}
\begin{itemize}
    \item Baseia-se na suposição de que pontos semelhantes estão próximos uns dos outros.
    \item A votação majoritária dos k vizinhos mais próximos determina o rótulo de classe.
    \item A distância euclidiana é uma medida de proximidade comumente utilizada.
\end{itemize}
\column{0.5\textwidth}
\textbf{Regressão:}
\begin{itemize}
    \item A média dos k vizinhos mais próximos é usada para fazer previsões.
    \item É similar à classificação, mas com valores contínuos.
\end{itemize}
\end{columns}
\end{frame}

% Slide 3: Características
\begin{frame}
\frametitle{Características}
\begin{itemize}
    \item Simples e fácil de entender: Ideal para iniciantes em machine learning.
    \item Versátil: Pode ser usado para classificação e regressão.
    \item Eficaz: Pode ser muito preciso em alguns casos.
    \item Preguiçoso: Armazena o conjunto de dados de treinamento e computa as previsões no momento da classificação.
\end{itemize}
\end{frame}

% Slide 4: Limitações
\begin{frame}
\frametitle{Limitações}
\begin{itemize}
    \item Escalabilidade: Pode ser lento e ineficiente com grandes conjuntos de dados.
    \item Sensibilidade ao ruído: Pode ser influenciado por valores inconsistentes nos dados.
    \item Escolha de k: O valor de k pode afetar significativamente o desempenho do modelo.
\end{itemize}
\end{frame}

% Slide 5: Aplicações
\begin{frame}
\frametitle{Aplicações}
\begin{itemize}
    \item Pré-processamento de dados: Imputação de valores ausentes.
    \item Motores de recomendação: Previsão de conteúdo de interesse para usuários.
    \item Finanças: Previsão de preços de ações e outras variáveis financeiras.
    \item Detecção de fraudes: Identificação de transações fraudulentas.
    \item Reconhecimento de imagem: Classificação de imagens.
\end{itemize}
\end{frame}

% Slide 6: Conclusões
\begin{frame}
\frametitle{Conclusões}
\begin{itemize}
    \item O KNN é um algoritmo fundamental em machine learning com diversas aplicações.
    \item Possui algumas limitações, mas é uma ferramenta útil para iniciantes e para problemas específicos.
\end{itemize}
\end{frame}

% Slide 7: Recursos Adicionais
\begin{frame}
\frametitle{Recursos Adicionais}
\begin{itemize}
    \item Link para a documentação do scikit-learn sobre o KNN: \url{https://scikit-learn.org/stable/modules/generated/sklearn.neighbors.KNeighborsClassifier.html}
    \item Artigo sobre o KNN em português: [URL inválido removido]
    \item Tutorial sobre o KNN com Python: [URL inválido removido]
\end{itemize}
\end{frame}

% Slide 8: Agradecimentos
\begin{frame}
\frametitle{Agradecimentos}
\begin{center}
\Huge Obrigado!
\end{center}
\end{frame}

\end{document}
